\documentclass[12pt,-letter paper]{article}

%\usepackage[left=1.5in,right=1in,top=1in,bottom=1in]{geometry}
%\usepackage[left=1.5in,right=1in]{geometry}
%\usepackage{geometry}
%\makeatletter%
%\textheight     243.5mm
%\textwidth      183.0mm
%\textwidth=31pc%
%\textheight=48pc
\usepackage{lipsum}% this package is included to get dummy paragraphs for sample purpose.
\usepackage{ulem}
\usepackage{alltt}
\usepackage{tfrupee}
\usepackage[anticlockwise,figuresright]{rotating}
\usepackage{pstricks}
\usepackage{wrapfig}
\usepackage{graphicx}
\usepackage{pstcol,pst-grad}
 \usepackage{bm}
\usepackage{enumitem}
\usepackage{circuitikz}
\usepackage{listings}
    \usepackage{color}                                            %%
    \usepackage{array}                                            %%
    \usepackage{longtable}                                        %%
    \usepackage{calc}                                             %%
    \usepackage{multirow}                                         %%
    \usepackage{hhline}                                           %%
    \usepackage{ifthen}                                           %%
  %optionally (for landscape tables embedded in another document): %%
    \usepackage{lscape}     
    \usepackage{gensymb}     
    \usepackage{tabularx}
\usepackage{ifthen}%
\usepackage{amsmath}%
\usepackage{color}%
\usepackage{float}%
\usepackage{graphicx}%
%\usepackage[right]{showlabels}%
\usepackage{boites}%
\usepackage{boites_exemples}%
\usepackage{graphicx,pstricks}
%\usepackage{enumerate}%
\usepackage{latexsym}
\usepackage[fleqn]{mathtools}
\usepackage{amssymb}
\usepackage{amssymb,amsfonts,amsthm}
\usepackage{mathrsfs,makeidx,listings,verbatim,moreverb}
\usepackage{siunitx}
%%\usepackage{amsthm,mathrsfs,makeidx,listings,verbatim,moreverb}
%\let\eqref\ref%  updated on 20th April 2017

\usepackage{hyperref}%
%\usepackage[dvips]{hyperref}%
\hypersetup{bookmarksopen=false}%
\usepackage{breakurl}%
\usepackage{tkz-euclide} % loads  TikZ and tkz-base
\DeclarePairedDelimiter\abs{\lvert}{\rvert}

\newcommand{\solution}{\noindent \textbf{Solution: }}
\providecommand{\mbf}{\mathbf}
\providecommand{\rank}{\text{rank}}
%\providecommand{\pr}[1]{\ensuremath{\Pr\left(#1\right)}}
\providecommand{\qfunc}[1]{\ensuremath{Q\left(#1\right)}}
\providecommand{\sbrak}[1]{\ensuremath{{}\left[#1\right]}}
\providecommand{\lsbrak}[1]{\ensuremath{{}\left[#1\right.}}
\providecommand{\rsbrak}[1]{\ensuremath{{}\left.#1\right]}}
\providecommand{\brak}[1]{\ensuremath{\left(#1\right)}}
\providecommand{\lbrak}[1]{\ensuremath{\left(#1\right.}}
\providecommand{\rbrak}[1]{\ensuremath{\left.#1\right)}}
\providecommand{\cbrak}[1]{\ensuremath{\left\{#1\right\}}}
\providecommand{\lcbrak}[1]{\ensuremath{\left\{#1\right.}}
\providecommand{\rcbrak}[1]{\ensuremath{\left.#1\right\}}}
\newenvironment{amatrix}[1]{%
  \left(\begin{array}{@{}*{#1}{c}|c@{}}
}{%
  \end{array}\right)
}
\theoremstyle{remark}
\newtheorem{rem}{Remark}
\newtheorem{theorem}{Theorem}[section]
\newtheorem{problem}{Problem}
\newtheorem{proposition}{Proposition}[section]
\newtheorem{lemma}{Lemma}[section]
\newtheorem{corollary}[theorem]{Corollary}
\newtheorem{example}{Example}[section]
\newtheorem{definition}[problem]{Definition}
\newcommand{\sgn}{\mathop{\mathrm{sgn}}}
%\providecommand{\abs}[1]{\left\vert#1\right\vert}
%\providecommand{\res}[1]{\Res\displaylimits_{#1}} 
%\providecommand{\norm}[1]{\left\lVert#1\right\rVert}
%\providecommand{\norm}[1]{\lVert#1\rVert}
\providecommand{\mtx}[1]{\mathbf{#1}}
%\providecommand{\mean}[1]{E\left[ #1 \right]}
\providecommand{\fourier}{\overset{\mathcal{F}}{ \rightleftharpoons}}
%\providecommand{\hilbert}{\overset{\mathcal{H}}{ \rightleftharpoons}}
\providecommand{\system}{\overset{\mathcal{H}}{ \longleftrightarrow}}
	%\newcommand{\solution}[2]{\textbf{Solution:}{#1}}
%\newcommand{\solution}{\noindent \textbf{Solution: }}
\newcommand{\cosec}{\,\text{cosec}\,}
\providecommand{\dec}[2]{\ensuremath{\overset{#1}{\underset{#2}{\gtrless}}}}
\newcommand{\myvec}[1]{\ensuremath{\begin{pmatrix}#1\end{pmatrix}}}
\newcommand{\myaugvec}[2]{\ensuremath{\begin{amatrix}{#1}#2\end{amatrix}}}
\newcommand{\mydet}[1]{\ensuremath{\begin{vmatrix}#1\end{vmatrix}}}
\newcommand\figref{Fig.~\ref}
\newcommand\appref{Appendix~\ref}
\newcommand\tabref{Table~\ref}
\newcommand{\romanNumeral}[1]{\uppercase\expandafter{\romannumeral#1}}
%\newcommand{\pr}[1]{\mathbb{P}(#1)}
%\numberwithin{equation}{section}
%\numberwithin{equation}{subsection}
%\numberwithin{problem}{section}
%\numberwithin{definition}{section}
%\makeatletter
%\@addtoreset{figure}{problem}
%\makeatother

%\let\StandardTheFigure\thefigure
\let\vec\mathbf
\def\inputGnumericTable{}                                 %%
%New macro definitions
\newcounter{matchleft}\newcounter{matchright}

\newenvironment{matchtabular}{%
  \setcounter{matchleft}{0}%
  \setcounter{matchright}{0}%
  \tabularx{\textwidth}{%
    >{\leavevmode\hbox to 1.5em{\stepcounter{matchleft}\arabic{matchleft}.}}X%
    >{\leavevmode\hbox to 1.5em{\stepcounter{matchright}\alph{matchright})}}X%
    }%
}{\endtabularx}
\title{GATE Question}
\date{\today}
\begin{document}
\maketitle
\begin{enumerate}
  
	\item Consider a $4$-bit counter constructed out of four flip-flops.It is formed by connecting the J and K inputs to logic high and feeding the Q output to the clock input of the following flip-flop (see the figure).The input signal to the counter is a series of square pulses and the change of state is triggered by the falling edge.At time t=t0 the outputs are in logic low state (Q0 = Q1 = Q2 = Q3 = 0).Then at t=t1,the logic state of the outputs is 
		\hfill(GATE-PH2020,30)
                \begin{figure}
                \centering
                \begin{circuitikz}
    \draw (2,2)rectangle(4,6);
    \draw (6,2)rectangle(8,6);
    \draw (10,2)rectangle(12,6);
    \draw (14,2)rectangle(16,6);
    \draw (0,0)node[left]{$1$}--(16,0);
    \draw (0,-1)node[right]{$logic high $};
     \draw (2,5.5)node[right]{$J$}--(1.5,5.5)--(1.5,0);
     \draw (6,5.5)node[right]{$J$}--(5.5,5.5)--(5.5,0);
     \draw (10,5.5)node[right]{$J$}--(9.5,5.5)--(9.5,0);
     \draw (14,5.5)node[right]{$J$}--(13.5,5.5)--(13.5,0);
     \draw (2,2.5)node[right]{$K$}--(1.5,2.5);
     \draw (6,2.5)node[right]{$K$}--(5.5,2.5);
     \draw (10,2.5)node[right]{$K$}--(9.5,2.5);
     \draw (14,2.5)node[right]{$K$}--(13.5,2.5);
     \draw (4,5.5)node[left]{$Q$}--(5,5.5)--(5,4)--(6,4)node[right]{$ck$};
     \draw (8,5.5)node[left]{$Q$}--(9,5.5)--(9,4)--(10,4)node[right]{$ck$};
     \draw (12,5.5)node[left]{$Q$}--(13,5.5)--(13,4)--(14,4)node[right]{$ck$};
     \draw (4.5,5.5)--(4.5,7.5)node[above]{$Q0$};
     \draw (8.5,5.5)--(8.5,7.5)node[above]{$Q1$};
     \draw (12.5,5.5)--(12.5,7.5)node[above]{$Q2$};
     \draw (16,5.5)node[left]{$Q$}--(16.5,5.5)--(16.5,7.5)node[above]{$Q3$};
     \draw (2,4)node[right]{$ck$}--(0.5,4)node[left]{$Input$};
         \draw (4,2.5)node[left]{$\overline{Q}$};
         \draw (8,2.5)node[left]{$\overline{Q}$};
         \draw (12,2.5)node[left]{$\overline{Q}$};
         \draw (16,2.5)node[left]{$\overline{Q}$};
         \draw(4.5,-4)--(5,-4)--(5,-3.5)--(5.5,-3.5)--(5.5,-4)--(6,-4)--(6,-3.5)--(6.5,-3.5)--(6.5,-4)--(7,-4)--(7,-3.5)--(7.5,-3.5)--(7.5,-4)--(8,-4)--(8,-3.5)--(8.5,-3.5)--(8.5,-4)--(9,-4)--(9,-3.5)--(9.5,-3.5)--(9.5,-4)-- (10,-4)--(10,-3.5)--(10.5,-3.5)--(10.5,-4)--(11,-4)--(11,-3.5)--(11.5,-3.5)--(11.5,-4)--(12,-4)--(12,-3.5)--(12.5,-3.5)--(12.5,-4)--(13,-4);
         \draw (8,-1) node[]{$ $4$-$bit ripple counter$ $};
         \draw[->] (5.5,-4.5)--(6.5,-4.5)node[right]{$t$};
         \draw (7,-5)node[right]{Input signal};
         \draw (4.75,-3.575)--(4.75,-4.6)node[right]{$t_0$};
         \draw (12.75,-3.575)--(12.75,-4.6)node[right]{$t_1$};
\end {circuitikz}

                \caption{Ripple Counter}
                \label{fig:4 Bit Ripple Counter}
                \end{figure}
		  \begin{enumerate}
          \item  Q0 = 1, Q1 = 0, Q2 = 0 and Q3 = 0 
          \item  Q0 = 0, Q1 = 0, Q2 = 0 and Q3 = 1
	  \item  Q0 = 1, Q1 = 0, Q2 = 1 and Q3 = 0
          \item  Q0 = 0, Q1 = 1, Q2 = 1 and Q3 = 1		  

      \end{enumerate}
  \end{enumerate}
  
\end{document}
