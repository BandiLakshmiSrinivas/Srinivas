\documentclass[2pt,-letter paper]{article}
\usepackage[left=1in, right=0.75in, top=1in, bottom=0.75in]{geometry}
\usepackage{graphicx} % Required for inserting images
\usepackage{siunitx}
\usepackage{setspace}
\usepackage{gensymb}
\usepackage{xcolor}
\usepackage{caption}
%\usepackage{subcaption}
\doublespacing
\singlespacing
\usepackage[none]{hyphenat}
\usepackage{amssymb}
\usepackage{relsize}
\usepackage[cmex10]{amsmath}
\usepackage{mathtools}
\usepackage{amsmath}
\usepackage{commath}
\usepackage{amsthm}
\interdisplaylinepenalty=2500
%\savesymbol{iint}
\usepackage{txfonts}
%\restoresymbol{TXF}{iint}
\usepackage{wasysym}
\usepackage{amsthm}
\usepackage{mathrsfs}
\usepackage{txfonts}
\let\vec\mathbf{}
\usepackage{stfloats}
\usepackage{float}
\usepackage{cite}
\usepackage{cases}
\usepackage{subfig}
%\usepackage{xtab}
\usepackage{longtable}
\usepackage{multirow}
%\usepackage{algorithm}
\usepackage{amssymb}
%\usepackage{algpseudocode}
\usepackage{enumitem}
\usepackage{mathtools}
%\usepackage{eenrc}
%\usepackage[framemethod=tikz]{mdframed}
\usepackage{listings}
%\usepackage{listings}
\usepackage[latin1]{inputenc}
%%\usepackage{color}{   
%%\usepackage{lscape}
\usepackage{textcomp}
\usepackage{titling}
\usepackage{hyperref}
%\usepackage{fulbigskip}   
\usepackage{tikz}
\usepackage{graphicx}
\lstset{
  frame=single,
  breaklines=true
}
\let\vec\mathbf{}
\usepackage{enumitem}
\usepackage{graphicx}
\usepackage{siunitx}
\let\vec\mathbf{}
\usepackage{enumitem}
\usepackage{graphicx}
\usepackage{enumitem}
\usepackage{tfrupee}
\usepackage{amsmath}
\usepackage{amssymb}
\usepackage{mwe} % for blindtext and example-image-a in example
\usepackage{wrapfig}
\graphicspath{{figs/}}
\providecommand{\mydet}[1]{\ensuremath{\begin{vmatrix}#1\end{vmatrix}}}
\providecommand{\myvec}[1]{\ensuremath{\begin{bmatrix}#1\end{bmatrix}}}
\providecommand{\cbrak}[1]{\ensuremath{\left\{#1\right\}}}
\providecommand{\brak}[1]{\ensuremath{\left(#1\right)}}
\title{MATHEMATICS}
\author{SECTION A}
\date{\today}
\begin{document}

\maketitle

\begin{enumerate}
\item If $\vec{A}$ is a square matrix satisfying $A'A = I$, write the value of $\mydet{A}$.
\item If $y=x\mydet{x}$, find $\dfrac{dy}{dx}$ for $x < 0$.
\item Find the order and degree (if defined) of the differential equation 
 \begin{align*}
 \dfrac{d^2y}{d^2x}+x\brak{\dfrac{dy}{dx}}^2=2x^2\log\brak{\dfrac{d^2y}{dx^2}}
 \end{align*}.
\item Fid the direction cosines of a line which makes equal angles with the coordinate axes.
 \item A line passes through the point with position vector $2\hat{i}-\hat{j}+4\hat{k}$ and is in the direction of the vector $\hat{i}+\hat{j}-2\hat{k}$. Find the equation of the line in cartesian form.
\author{SECTION B}
\item Examine whether the operation $*$ defined on $\vec{R}$, the set of all real numbers, by $a * b = \sqrt{a^2+b^2}$ is a binary operation or not, and if it is a binary operation, find whether it is associative or not.
\item Find: \begin{align*}\int\sqrt{3-2x-x^2}dx\end{align*}
\item Find: \begin{align*}\int{\frac{\sin^3{x}+\cos^3{x}}{\sin^2{x}\cos^2{x}}}dx\end{align*}
\item Find: \begin{align*}\int{\frac{x-3}{\brak{x-1}^3}}e^xdx\end{align*}
\item If ${\overrightarrow{\mydet{a}}}$ = $2, {\overrightarrow{\mydet{b}}}$ = $7$ and $\overrightarrow{a}\times\overrightarrow{b}$ = $3\hat{i}+2\hat{j}+6\hat{k}$, find the angle between $\overrightarrow{a}$ and $\overrightarrow{b}$.
\item Find the volume of a cuboid whose edges are given by $-3\hat{i}+7\hat{j}+5\hat{k}$,$-5\hat{i}+7hat{j}-3\hat{k}$ and $7\hat{i}-5\hat{j}-3\hat{k}$.
\item Find the probability distribution of X, the number of heads in a simultaneous toss of two coins.
\item Find the value of $\sin\brak{\cos^{-1}{\frac{4}{5}}+{\tan^{-1}{\frac{2}{3}}}}$.
\item If $\brak{\cos x}^y = \brak{\sin y }^x$, find $\dfrac{dy}{dx}$.
\item  Find the equation of the normal to the curve ${x}^2 = 4y$ which passes through the point $\myvec{-1,4}$. 
\item Solve the differential equation : 
\begin{align*}
\dfrac{dy}{dx}= \sqrt{\brak{\frac{x+y\cos x}{1+\sin x}}}
\end{align*}
\item The scalar product of the vector $\overrightarrow{a} = \hat{i}+\hat{j}+\hat{k}$ with a unit vector along the sum of the vectors $\overrightarrow{b} = 2\hat{i}+4\hat{j}-5\hat{k}$ and $\overrightarrow{c} = \hat{\lambda}+2\hat{j}+3\hat{k}$ is equal to $1$. Find the value of $\lambda$ and hence find the unit vector along $\overrightarrow{b}+\overrightarrow{c}$.
\item If the lines $\frac{x-1}{-3}=\frac{y-2}{2\lambda}=\frac{z-3}{2}$ and $\frac{x-1}{3\lambda}=\frac{y-1}{2}=\frac{z-6}{-5}$ are perpendicular, find the value of $\lambda$. Hence find weather the lines are intersecting or not.
\item If ${\vec{A}} = \myvec{1&3&4\\2&1&1\\5&1&1}$, find $A^{-1}$.
        Hence solve the system of equations 
		\begin{align*}
                {x+3y+4z}=8 \\
                {2x+y+2z}=5 \\
            \text{and}\hspace{6pt} {5x+y+z}=7
            \end{align*} 
\item  Show that the height of the cylinder of maximum volume that can be inscribed in a sphere of radius $R$ is $\frac{2R}{\sqrt{3}}$. Also find the maximum volume.
 \item Using integration, find the area of the triangular region whose sides have the equations ${y}={2x}+1$, ${y}={3x}+1$ and ${x}=4$.
\item  A company produces two types of goods, $A$ and $B$, that require gold and silver. Each unit of type $A$ requires $3$ g of silver and $1$ g of gold while that of type $B$ requires $1$ g of silver and $2$ g of gold. The company can use at the most $9$ g of silver and $8$ g of gold. If each unit of type $A$ brings a profit of \rupee $40$ and that of type $B$ \rupee $50$, find the number of units of each type that the company should produce to maximize profit. Formulate the above LPP and solve it graphically and also find the maximum profit.
\end{enumerate}
\end{document}`
