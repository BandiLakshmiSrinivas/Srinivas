\documentclass[2pt,-letter paper]{article}
\usepackage[left=1in, right=0.75in, top=1in, bottom=0.75in]{geometry}
\usepackage{graphicx} % Required for inserting images
\usepackage{siunitx}
\usepackage{setspace}
\usepackage{gensymb}
\usepackage{xcolor}
\usepackage{caption}
%\usepackage{subcaption}
\doublespacing
\singlespacing
\usepackage[none]{hyphenat}
\usepackage{amssymb}
\usepackage{relsize}
\usepackage[cmex10]{amsmath}
\usepackage{mathtools}
\usepackage{amsmath}
\usepackage{commath}
\usepackage{amsthm}
\interdisplaylinepenalty=2500
%\savesymbol{iint}
\usepackage{txfonts}
%\restoresymbol{TXF}{iint}
\usepackage{wasysym}
\usepackage{amsthm}
\usepackage{mathrsfs}
\usepackage{txfonts}
\let\vec\mathbf{}
\usepackage{stfloats}
\usepackage{float}
\usepackage{cite}
\usepackage{cases}
\usepackage{subfig}
%\usepackage{xtab}
\usepackage{longtable}
\usepackage{multirow}
%\usepackage{algorithm}
\usepackage{amssymb}
%\usepackage{algpseudocode}
\usepackage{enumitem}
\usepackage{mathtools}
%\usepackage{eenrc}
%\usepackage[framemethod=tikz]{mdframed}
\usepackage{listings}
%\usepackage{listings}
\usepackage[latin1]{inputenc}
%%\usepackage{color}{   
%%\usepackage{lscape}
\usepackage{textcomp}
\usepackage{titling}
\usepackage{hyperref}
%\usepackage{fulbigskip}   
\usepackage{tikz}
\usepackage{graphicx}
\lstset{
  frame=single,
  breaklines=true
}
\let\vec\mathbf{}
\usepackage{enumitem}
\usepackage{graphicx}
\usepackage{siunitx}
\let\vec\mathbf{}
\usepackage{enumitem}
\usepackage{graphicx}
\usepackage{enumitem}
\usepackage{tfrupee}
\usepackage{amsmath}
\usepackage{amssymb}
\usepackage{mwe} % for blindtext and example-image-a in example
\usepackage{wrapfig}
\graphicspath{{figs/}}
\providecommand{\cbrak}[1]{\ensuremath{\left\{#1\right\}}}
\providecommand{\brak}[1]{\ensuremath{\left(#1\right)}}
\newcommand{\sgn}{\mathop{\mathrm{sgn}}}
\providecommand{\abs}[1]{\left\vert#1\right\vert}
\providecommand{\res}[1]{\Res\displaylimits_{#1}} 
\providecommand{\norm}[1]{\left\lVert#1\right\rVert}
%\providecommand{\norm}[1]{\lVert#1\rVert}
\providecommand{\mtx}[1]{\mathbf{#1}}
\providecommand{\mean}[1]{E\left[ #1 \right]}
\providecommand{\fourier}{\overset{\mathcal{F}}{ \rightleftharpoons}}
%\providecommand{\hilbert}{\overset{\mathcal{H}}{ \rightleftharpoons}}
\providecommand{\system}{\overset{\mathcal{H}}{ \longleftrightarrow}}
 %\newcommand{\solution}[2]{\textbf{Solution:}{#1}}
%\newcommand{\solution}{\noindent \textbf{Solution: }}
\newcommand{\cosec}{\,\text{cosec}\,}
\providecommand{\dec}[2]{\ensuremath{\overset{#1}{\underset{#2}{\gtrless}}}}
\newcommand{\myvec}[1]{\ensuremath{\begin{pmatrix}#1\end{pmatrix}}}
\newcommand{\myaugvec}[2]{\ensuremath{\begin{amatrix}{#1}#2\end{amatrix}}}
\newcommand{\mydet}[1]{\ensuremath{\begin{vmatrix}#1\end{vmatrix}}}
\title{MATHEMATICS}
\author{SECTION A}
\date{\today}
\begin{document}

\maketitle

\begin{enumerate}
\section{Matrix}
\item If ${A}$ is a square matrix satisfying ${A'A}= I$, write the value of $\mydet{A}$.
\item Using properties of determinants, show that 
\begin{align*}
\mydet{3a & -a+b & -a+c \\ -b+a & 3b & -b+c \\ -c+a & -c+b & 3c} = 3\brak{a+b+c}\brak{ab+bc+ca}
\end{align*}	
\section{Differentiation}
\item If $x\sqrt{1+y}$+ $y\sqrt{1+x}$=0 and $x\neq y$, prove that $\dfrac{dy}{dx} = -\frac{1}{\brak{x+1}^2}$.
\item If $y=x\mydet{x}$, find $\dfrac{dy}{dx}$ for $x < 0$.
\item Solve the differential equation: 
\begin{align*}
{x}\dfrac{dy}{dx}= {y}-{x}\tan\brak{\frac{y}{x}}
\end{align*}
\item If $\brak{\cos x}^y = \brak{\sin y }^x$, find $\dfrac{dy}{dx}$.
\item Find the differential equation representing the family of curves ${y}=ae^{2x}+5$, where $a$ is an arbitrary constant.
\item If ${x}=ae^t\brak{\sin{t}+\cos{t}}$ and ${y}=ae^t\brak{\sin{t}-\cos{t}}$, then prove that $\dfrac{dy}{dx}=\frac{x+y}{x-y}$.
\section{Integration}
\item Find: \begin{align*}\int\sqrt{3-2x-x^2}dx\end{align*}
\item Find: \begin{align*}\int{\frac{x^2+x+1}{\brak{x+2}\brak{x^2+1}}}dx\end{align*}
\item Find: 
 \begin{align*}
 \int{\frac{x-5}{\brak{x-3}^3}}e^x dx
 \end{align*}
\item Find: 
\begin{align*}
\int{\frac{2\cos x}{\brak{1-\sin x}\brak{2-\cos^2 x}}}dx
\end{align*}  
\section{Vectors}
\item A line passes through the point with position vector $2\hat{i}-\hat{j}+4\hat{k}$ and is in the direction of the vector $\hat{i}+\hat{j}-2\hat{k}$. Find the equation of the line in cartesian form.
\item Find the volume of a cuboid whose edges are given by $-3\hat{i}+7\hat{j}+5\hat{k}$,$-5\hat{i}+7\hat{j}-3\hat{k}$ and $7\hat{i}-5\hat{j}-3\hat{k}$.
\item The scalar product of the vector $\overrightarrow{a} = \hat{i}+\hat{j}+\hat{k}$ with a unit vector along the sum of the vectors $\overrightarrow{b} = 2\hat{i}+4\hat{j}-5\hat{k}$ and $\overrightarrow{c} = \hat{\lambda}+2\hat{j}+3\hat{k}$ is equal to $1$. Find the value of $\lambda$ and hence find the unit vector along $\overrightarrow{b}+\overrightarrow{c}$.
\item Show that the points $A\brak{-2\hat{i}+3\hat{j}+5\hat{k}}$, $B\brak{\hat{i}+2\hat{j}+3\hat{k}}$ and $C\brak{7\hat{i}-\hat{k}}$ are collinear.
\item Find the direction cosines of a line which makes equal angles with the coordinate axes.
\item If the lines $\frac{x-1}{-3}=\frac{y-2}{2\lambda}=\frac{z-3}{2}$ and $\frac{x-1}{3\lambda}=\frac{y-1}{2}=\frac{z-6}{-5}$ are perpendicular, find the value of $\lambda$. Hence find weather the lines are intersecting or not.
\item If ${\overrightarrow{\mydet{a}}}$ = $2, {\overrightarrow{\mydet{b}}}$ = $7$ and $\overrightarrow{a}\times\overrightarrow{b}$ = $3\hat{i}+2\hat{j}+6\hat{k}$, find the angle between $\overrightarrow{a}$ and $\overrightarrow{b}$.
\section{Probability}
\item A bag contains $5$ red and $4$ black balls, a second bag contains $3$ red and $6$ black balls. One of the two bags is selected at random and two balls are drawn at random (without replacement) both of which are found to be red. Find the probability that the balls are drawn from the second bag.
\item Find the probability distribution of $X$, the number of heads in a simultaneous toss of two coins.
\item There are three coins. One is a two-headed coin, another is a biased coin that comes up heads $75\%$ of the time and the third is an unbiased coin. One of the three coins is chosen at random and tossed. If it shows heads, what is the probability that it is the two-headed coin ?
\section {Functions}
\item Check whether the relation $R$ defined on the set $A=\cbrak{1,2,3,4,5,6}$ as $R =\cbrak {(a, b) : b = a + 1}$ is reflexive, symmetric or transitive.
\item Let $f$ : $N\rightarrow Y $ be a function defined as $f\brak{x}= 4x + 3$, where $Y=\cbrak{y\in N:y=4x+3, \text{ for some } x\in N}$. Show that $f$ is invertible. Find its inverse.
\section{Intersection of coins}
\item Find the equation of the normal to the curve ${x}^2 = 4y$ which passes through the point $\myvec{-1,4}$.
\item Find the point on the curve ${y}^2 = 4x$, which is nearest to the point $\myvec{2,-8}$.
\section{Optimization}
\item Show that the height of the cylinder of maximum volume that can be inscribed in a sphere of radius $R$ is $\frac{2R}{\sqrt{3}}$. Also find the maximum volume.
\item The volume of a cube is increasing at the rate of $8 cm^3/s$. How fast is the surface area increasing when the length of its edge is $12 cm$ ?
\section{Algebra}
\item Solve for x:
\begin{align*}
\tan^{-1}\brak{x+1}+\tan^{-1}\brak{x-1}=\tan^{-1}\brak{\frac{8}{31}}
\end{align*}
\end{enumerate}
\end{document}
